\label{chap:dialects}

The Norwegian language presents an interesting case for dialectologists in that there does not exist a standard version of the language.
While there are two written languages,%
\footnote{I use Bokm\aa{}l for the Norwegian examples in this chapter, as this is the written language used in the ScanDiaSyn corpus.}
neither is representative of any one dialect, and adopting a different dialect, especially Urban East Norwegian (spoken in and around Oslo), is considered inauthentic and looked down upon.

This section is structured as follows:
I first introduce the classification that is typically applied to Norwegian dialectology (\autoref{sec:dialects-dialectology}).
I then present the data that I work with (\autoref{sec:dialects-data}).
In \autoref{sec:dialects-dialectometry}, I introduce previous approaches to automatic dialect classification, and in \autoref{sec:norwegian-dialectometry}, I present prior dialectometric work with Norwegian data.
Next, I explain the classification approach (\autoref{sec:dialects-method}).
In \autoref{sec:dialects-results}, I show and analyze the results, including general observations on the LIME scores, as well as an analysis of how the linguistic features most commonly used for dividing the Norwegian dialect landscape are (not) represented in the results, and finally other recurrent linguistic features in the results.
