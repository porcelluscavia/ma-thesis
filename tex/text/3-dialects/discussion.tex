\subsection{Discussion}
\label{sec:dialects-discussion}

Many of the features that got assigned high importance scores by LIME serve as examples for the linguistic patterns described by dialectologists.
However, not all features that are important for the label prediction are easy to understand for humans or fall into easily recognizable feature categories.
Additionally, many of the features with high-importance scores showcase linguistic traits that are not often discussed in Norwegian dialectology, such as the different variants of \textit{noe(n)} `some, somebody, something' or the past participle endings in North Norwegian.

The features that have high importance scores for a dialect group are not always very representative of the entire group (although these exist, such as the West Norwegian /(r)rs/), but sometimes only represent characteristic traits of the dialects spoken in one subregion (e.g. the diphthongs /ao, \aa{}o/ in some parts of West Norway).
This also results in there sometimes being several seemingly contradictory features that have high importance scores for the same label, such as the West Norwegian first person singular variants /i/, /ei/ and /eg/ that are all among the 50 highest-ranking features for that dialect group.
It would be interesting to explore how this might change if the number of input features is restricted further and relatively infrequent features are excluded.

It might also be insightful to examine the importance scores for features that are encoded differently, for instance as sound correspondences between the dialects and a reference doculect.

Furthermore, it would me worthwhile to explore which features have high importance scores and are common in false positives/negatives: are there patterns as to which linguistic features lead the classifier astray?
